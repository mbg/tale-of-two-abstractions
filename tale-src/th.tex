\section{Template Haskell}
\label{sec:th}

\textbf{COPIED FROM OLD VERSION}

We have implemented our system in a Haskell library\footnote{\url{https://github.com/mbg/monadic-state-hierarchies}} which primarily provides a $\mathit{QuasiQuoter}$ \citep{mainland2007s} for our object notation. Multiple state declarations may be included in a single quotation and, indeed, this is necessary if multiple state classes depend on each other. The examples in Section \ref{sec:usage} can be used exactly as shown. For example, for $\mathit{MListItem}$ we write:
\begin{displaymath}
\begin{array}{l}
[\mathit{state}\mid \mathbf{state}~\mathit{MListItem}~a~\mathbf{where} \\
\quad \begin{array}{lcl}
\mathbf{data}~\mathit{val} & :: & a \\
\mathbf{data}~\mathit{next}  & :: & \mathit{Maybe}~(\mathit{MListItem}~a)
\end{array}\\\\
\quad \begin{array}{lcl}
\mathit{insertAtEnd} & :: & \mathit{MListItem}~a \to ()\\
\mathit{insertAtEnd}~v & = & \mathbf{do} \\
\multicolumn{3}{l}{\quad \begin{array}{l}
\mathit{switch}~\mathit{next}~\$~\lambda case \\
\quad \begin{array}{lcl}
\mathit{Nothing} & \to & \mathit{next} <: \mathit{Just}~v  \\
(\mathit{Just}~n) & \to & \mathbf{do} \\
\multicolumn{3}{l}{\quad \mathit{next} <: \mathit{Just}~(\mathit{object}~(n.!\mathit{insertAtEnd})~v) }
\end{array}
\end{array}}
\end{array}\\\mid]
\end{array}
\end{displaymath}
The quotation is parsed and then translated into an abstract representation of Haskell's syntax, according to the rules shown in Section \ref{sec:auto}. Method typings must be given, similar to type classes. We parse them as standard Haskell, but their co-domains will be wrapped into \emph{e.g.} $\mathit{MListItemM}$ in the example above. Method equations are also parsed as standard Haskell, but there is no need to apply any transformations. 

In order to allow programmers to work with the object classes, our library contains a number of combinators. Most of these combinators operate on \emph{selectors} which are defined as:
\begin{displaymath}
\begin{array}{lcl}
\mathbf{data}~\mathit{Selector}~o~s~m~a~b & = & \mathit{MkMethod}~\{ \\
\multicolumn{3}{l}{\quad \begin{array}{lcl}
\mathit{mInternal} & :: & \mathit{a} \to \mathit{StateT}~s~m~b, \\
\mathit{mExternal} & :: & o \to a \to m~(b,o)
\end{array}} \\
\multicolumn{3}{l}{\} \mid \mathit{MkField}~\{} \\
\multicolumn{3}{l}{\quad \begin{array}{lcl}
\mathit{fExGet} & :: & o \to m~(b,o), \\
\mathit{fInGet} & :: & \mathit{StateT}~s~m~b, \\
\mathit{fExSet} & :: & o \to a \to m~((),o), \\
\mathit{fInSet} & :: & a \to \mathit{StateT}~s~m~b
\end{array}} \\
\multicolumn{3}{l}{\} \mid \mathit{MkCall}~\{} \\
\multicolumn{3}{l}{\quad \begin{array}{lcl}
\mathit{cCall} & :: & a \to (b,o)
\end{array}} \\
\}
\end{array}
\end{displaymath}
Intuitively, a selector is a record of all functions belonging to a field or method. The $\mathit{MkCall}$ constructor is used to represent selectors which have been applied to an object. For convenience, we define a type alias for fields since their argument and return types will always be the same:
\begin{displaymath}
\begin{array}{lcl}
\mathbf{type}~\mathit{Field}~o~s~m~a & = & \mathit{Selector}~o~s~m~a~a
\end{array}
\end{displaymath}
We enrich the type classes that are generated for object classes with selectors for all fields and methods, using the original names from the state declaration. For example, $\mathit{MListItemLike}$ contains
\begin{displaymath}
\begin{array}{l}
\mathit{val} :: \mathit{MListItemLike}~o~s~m~a \Rightarrow \\
\qquad \mathit{Field}~(o~a)~(s~a)~m~a \\
\mathit{next} ::  \mathit{MListItemLike}~o~s~m~a \Rightarrow \\
\qquad  \mathit{Field}~(o~a)~(s~a)~m~(\mathit{Maybe}~(\mathit{MListItem}~a))\\
\mathit{insertAtEnd} ::  \mathit{MListItemLike}~o~s~m~a \Rightarrow \\
\qquad  \mathit{Selector}~(o~a)~(s~a)~m~(\mathit{MListItem}~a)~()\\
\end{array}
\end{displaymath}
in addition to the type class methods shown in Section \ref{sec:auto}. This allows us to use the field and method names directly in any context without any compile-time transformations which resolve them to the appropriate functions. For example, in the instance for $\mathit{MListItem}$, $\mathit{next}$ has the following value:
\begin{displaymath}
\mathit{next} = \mathit{MkField} \mathit{\_get\_next} \mathit{\_get\_next}' \mathit{\_set\_next} \mathit{\_set\_next}'
\end{displaymath}
When used in \emph{e.g.} the $<:$ operator, the appropriate setter is extracted from the record and applied to the new value for the field:
\begin{displaymath}
\begin{array}{l}
(<:)  ::  \mathit{Monad}~m \Rightarrow \mathit{Field}~o~s~m~\mathit{val} \to \mathit{val} \to \mathit{StateT}~s~m~()\\
(\mathit{MkField}~\_~\_~\_~s') <: val  =  s val
\end{array}
\end{displaymath}
Objects and selectors can be combined using the $.!$ operator which in turn constructs a new selector which can then be used by one of the combinators. The $.!$ is part of the $\mathit{Object}$ type class which in our implementation is the super class of all objects, instead of $\mathit{Monad}$, although $\mathit{Monad}$ is still the superclass of $\mathit{Object}$:
\begin{displaymath}
\begin{array}{l}
\mathbf{class}~\mathit{Monad}~m \Rightarrow \mathit{Object}~\mathit{obj}~m~\mathbf{where}\\
\quad (.!) :: \mathit{obj} \to \mathit{Selector}~\mathit{obj}~s~m~\mathit{arg}~\mathit{ret} \to \\ \qquad \mathit{Selector}~\mathit{obj}~s~m~\mathit{arg}~\mathit{ret}
\end{array}
\end{displaymath}
There is one instance of this class for all base object classes:
\begin{displaymath}
\begin{array}{l}
\mathbf{instance}~\mathit{Object}~a~\mathit{Identity}~\mathbf{where}\\
\quad \begin{array}{lcl}
\mathit{obj}~.!~(\mathit{MkMethod}~\mathit{int}~\mathit{ext}) & = & \\
\multicolumn{3}{l}{\quad \mathit{MkCall}~ \$~\lambda \mathit{arg} \to \mathit{runIdentity}~\$~\mathit{ext}~\mathit{obj}~\mathit{arg} } %\\
%\mathit{obj}~.!~(\mathit{MkField}~g~\_~s~\_) & = & \\
% \multicolumn{3}{l}{\quad \mathit{MkCall}~\$~\mathit{const}~\$~\mathit{runIdentity}~\$ ~g~\mathit{obj}}
\end{array}
\end{array}
\end{displaymath}
For example, if an object is combined with a selector, then we create a wrapper function around the method after applying it to the object. $\mathit{MkCall}$ values are then used by \emph{e.g.} the $\mathit{object}$ and $\mathit{result}$ combinators as follows:
\begin{displaymath}
\begin{array}{l}
\mathit{object} :: \mathit{Selector}~o~s~\mathit{Identity}~\mathit{arg}~\mathit{ret} \to arg \to o\\
\mathit{object}~(\mathit{MkCall}~\mathit{call})  = \mathit{snd} \circ \mathit{call}
\end{array}
\end{displaymath}

