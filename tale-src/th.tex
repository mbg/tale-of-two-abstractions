\section{Template Haskell}
\label{sec:th}

We have developed a Haskell library\footnote{\url{https://github.com/mbg/monadic-state-hierarchies} on GitHub and \url{http://hackage.haskell.org/package/msh} on Hackage} which implements the automatic translation from a simple object-oriented language to the encoding shown in Section \ref{sec:auto}. Object classes can be defined inside of a Haskell source file with the help of a Template Haskell $\mathit{QuasiQuoter}$\cite{mainland2007s}. The example for a simple expression language we have shown in Section \ref{sec:encoding} using Java syntax can easily be converted into our state declarations:
\begin{displaymath}
\begin{array}{l}
[\mathit{state}\mid\\\mathbf{abstract}~\mathbf{state}~\mathit{Expr}~\mathbf{where} \\
\quad \begin{array}{lcl}
\mathit{eval} & :: & \mathit{Int}\\
\end{array}\\\\
\mathbf{state}~\mathit{Add} : \mathit{Expr}~\mathbf{where} \\
\quad \begin{array}{lcl}
\mathbf{data}~\mathit{left} & :: & \mathit{Expr} \\
\mathbf{data}~\mathit{right}  & :: & \mathit{Expr}
\end{array}\\\\
\quad \begin{array}{lcl}
\mathit{eval} & :: & \mathit{Int}\\
\mathit{eval} & = & \mathbf{do} \\
\multicolumn{3}{l}{\quad  \begin{array}{lcl}
x & \leftarrow & \mathit{this}.!\mathit{left}.!\mathit{eval}\\
y & \leftarrow & \mathit{this}.!\mathit{right}.!\mathit{eval}\\
\multicolumn{3}{l}{\mathit{return}~(x + y) }
\end{array}} 
\end{array}\\
\mid]
\end{array}
\end{displaymath}
The quotation is parsed and then translated into an abstract representation of Haskell's syntax, according to the rules shown in Section \ref{sec:auto}. Method typings must be given, similar to type classes. We parse them as standard Haskell, but their co-domains will be wrapped into \emph{e.g.} $\mathit{AddM}$ in the example above. Method equations are also parsed as normal Haskell, but no transformations are applied. The code for this and other examples may be found in our code repository.

\subsection{Selectors}

In order to allow programmers to work with the object classes, our library contains a number of combinators. Most of these combinators operate on \emph{selectors} which are defined using a GADT and are indexed over whether they represent a mutable member, \emph{i.e.} a field, or an immutable member, \emph{i.e.} a method. The $\mathit{MemberType}$ type is promoted to the kind level and the $\mathit{Mutable}$ and $\mathit{Immutable}$ constructors are used as types:
\begin{displaymath}
\begin{array}{l}
\mathbf{data}~\mathit{MemberType} = \mathit{Mutable} \mid \mathit{Immutable} \\\\
\mathbf{data}~\mathit{Selector}~(\mathit{ty} :: \mathit{MemberType})~o~s~m~a~\mathbf{where} \\
\quad \begin{array}{lcl}
 \mathit{MkMethod} & :: & \mathit{StateT}~s~m~a \to \\
                   &    & (o \to m~(a, o)) \to\\
                   &    & \mathit{Selector}~\mathit{Immutable}~o~s~m~a \\
\mathit{MkField}   & :: & (o \to m (a,o)) \to \\
                   &    & \mathit{StateT}~s~m~a \to \\
                   &    & (o \to a \to m~((),o)) \to \\
                   &    & (a \to \mathit{StateT}~s~m~()) \to \\
                   &    & \mathit{Selector}~\mathit{Mutable}~o~s~m~a
\end{array}
\end{array}
\end{displaymath}
Intuitively, a selector is a value which contains both, internal and external, versions of a member so that the same name can be used for both. For fields, it contains both versions for the getter and the setter. 
The type classes which are generated for object classes contain selectors for all of their members, using the original names from the state declaration. For example, including inherited members and in addition to the type class methods shown in Section \ref{sec:auto}, the $\mathit{Add}$ object class contains:
\begin{displaymath}
\begin{array}{l}
\mathit{left} :: \mathit{AddLike}~o~s~m \Rightarrow \\
\qquad \mathit{Selector}~\mathit{Mutable}~o~s~m~\mathit{Expr} \\
\mathit{right} :: \mathit{AddLike}~o~s~m \Rightarrow \\
\qquad \mathit{Selector}~\mathit{Mutable}~o~s~m~\mathit{Expr} \\
\mathit{eval} :: \mathit{EvalLike}~o~s~m \Rightarrow \\
\qquad \mathit{Selector}~\mathit{Immutable}~o~s~m~\mathit{Int} \\
\end{array}
\end{displaymath}
Selectors allow us to use member names directly in either an internal or external call context without any compile-time transformations which resolve them to the appropriate functions. For example, in the type class instance for $\mathit{Add}$, $\mathit{left}$ has the following value:
\begin{displaymath}
\mathit{left} = \mathit{MkField}~\mathit{\_ext\_get\_next}~\mathit{\_int\_get\_next}~\mathit{\_ext\_set\_next}~\mathit{\_int\_set\_next}
\end{displaymath}

\subsection{Selector composition}

In object-oriented languages, member access can be chained together using the $.$ ``operator''. For example, $\mathit{obj}.\mathit{foo}.\mathit{bar}$ gets the value of a field $\mathit{foo}$ belonging to an object $\mathit{obj}$ and then invokes the $\mathit{bar}$ method on the value of $\mathit{foo}$. We have a similar mechanism in our library. However, since $.$ is already used for function composition, we use $.!$ instead. This operator is part of a type class $\mathit{Object}$, which is added as a super-class constraint to every type class generated for the interface of a base class:
\begin{displaymath}
\begin{array}{l}
\mathbf{infixr}~8~.!\\
\mathbf{class}~\mathit{Monad}~m \Rightarrow \mathit{Object}~\mathit{obj}~\mathit{st}~m~\mathit{where} \\
\qquad \begin{array}{lcl}
\mathit{this} & :: & \mathit{This}~\mathit{obj}~\mathit{st}~\mathit{m}~\mathit{obj} \\
\mathit{this} & = & \mathit{MkThis} \\\\

(.!) & :: & \forall r~\mathit{ty}.\\ 
     &    & \mathit{obj} \to \\
     &    & \mathit{Selector}~\mathit{ty}~(\mathit{QueryObject}~\mathit{obj})~\mathit{st}~(\mathit{QueryMonad}~\mathit{obj}~\mathit{m})~r \to  \\
     &    & \mathit{QueryResult}~\mathit{obj}~\mathit{ty}~\mathit{st}~m~r
\end{array}
\end{array}
\end{displaymath}
Note that the $\mathit{This}$ type is defined as $\mathbf{data}~\mathit{This}~o~s~m~a$ and simply serves to capture the type arguments of instances of the $\mathit{Object}$ type class. The type of $.!$ is highly flexible. The left argument of the operator can be any type that is made an instance of the $\mathit{Object}$ class. Primarily, those are objects as the name would imply, but selectors and $\mathit{This}$ are also instances of $\mathit{Object}$. The right argument is always a selector. We use the $\mathit{QueryObject}$ and $\mathit{QueryMonad}$ type families to determine what object the selector will be accessing and what monad stack it will run on, depending on the left argument of $.!$.

The type of selector we obtain from composing two selectors is determined by a closed type function:
\begin{displaymath}
\begin{array}{l}
\mathbf{type}~\mathbf{family}~\mathit{MemberComposeResult} \\
\quad (\mathit{lhs} :: \mathit{MemberType})~(\mathit{rhs} :: \mathit{MemberType} :: \mathit{MemberType})~\mathbf{where} \\\\
\qquad \begin{array}{lcl}
\mathit{MemberComposeResult}~\mathit{Immutable}~\mathit{Immutable} & = & \mathit{Immutable} \\
\mathit{MemberComposeResult}~\mathit{Immutable}~\mathit{Mutable} & = & \mathit{Immutable} \\
\mathit{MemberComposeResult}~\mathit{Mutable}~\mathit{Immutable} & = & \mathit{Immutable} \\
\mathit{MemberComposeResult}~\mathit{Mutable}~\mathit{Mutable} & = & \mathit{Mutable} 
\end{array}
\end{array}
\end{displaymath}

In most object-oriented languages, $\mathit{null}$ is automatically a value of all reference types. Members of a field or variable can be accessed without checking whether the value is $\mathit{null}$ or not, frequently leading to runtime errors which are difficult to track down. Functional languages, such as ML or Haskell, do not have $\mathit{null}$ values -- they have a better solution: the option type. In Haskell, this type is called $\mathit{Maybe}$ and is defined as:
\begin{displaymath}
\begin{array}{lcl}
\mathbf{data}~\mathit{Maybe}~a & = & \mathit{Nothing} \mid \mathit{Just}~a
\end{array}
\end{displaymath}
In other words, a value of type $\mathit{Maybe}~a$ is either $\mathit{Nothing}$ -- the rough equivalent of $\mathit{null}$ -- or $\mathit{Just}~x$ for some $x :: a$. Case analysis must be performed on every such value to determine whether it contains a value of type $a$ or not. Functions expecting an argument of type $a$ cannot be applied to a value of type $\mathit{Maybe}~a$. Object-oriented languages have recently begun to adapt this type as a type-safe alternative in favour of $\mathit{null}$. However, checking whether a value is $\mathit{Nothing}$ or not can still be tedious. 

A second composition operator

\subsection{Combinators}

When used in \emph{e.g.} the $<:$ operator, the appropriate setter is extracted from the record and applied to the new value for the field:
\begin{displaymath}
\begin{array}{l}
(<:)  ::  \mathit{Monad}~m \Rightarrow \mathit{Field}~o~s~m~\mathit{val} \to \mathit{val} \to \mathit{StateT}~s~m~()\\
(\mathit{MkField}~\_~\_~\_~s') <: val  =  s val
\end{array}
\end{displaymath}
Objects and selectors can be combined using the $.!$ operator which in turn constructs a new selector which can then be used by one of the combinators. The $.!$ is part of the $\mathit{Object}$ type class which in our implementation is the super class of all objects, instead of $\mathit{Monad}$, although $\mathit{Monad}$ is still the superclass of $\mathit{Object}$:
\begin{displaymath}
\begin{array}{l}
\mathbf{class}~\mathit{Monad}~m \Rightarrow \mathit{Object}~\mathit{obj}~m~\mathbf{where}\\
\quad (.!) :: \mathit{obj} \to \mathit{Selector}~\mathit{obj}~s~m~\mathit{arg}~\mathit{ret} \to \\ \qquad \mathit{Selector}~\mathit{obj}~s~m~\mathit{arg}~\mathit{ret}
\end{array}
\end{displaymath}
There is one instance of this class for all base object classes:
\begin{displaymath}
\begin{array}{l}
\mathbf{instance}~\mathit{Object}~a~\mathit{Identity}~\mathbf{where}\\
\quad \begin{array}{lcl}
\mathit{obj}~.!~(\mathit{MkMethod}~\mathit{int}~\mathit{ext}) & = & \\
\multicolumn{3}{l}{\quad \mathit{MkCall}~ \$~\lambda \mathit{arg} \to \mathit{runIdentity}~\$~\mathit{ext}~\mathit{obj}~\mathit{arg} } %\\
%\mathit{obj}~.!~(\mathit{MkField}~g~\_~s~\_) & = & \\
% \multicolumn{3}{l}{\quad \mathit{MkCall}~\$~\mathit{const}~\$~\mathit{runIdentity}~\$ ~g~\mathit{obj}}
\end{array}
\end{array}
\end{displaymath}
For example, if an object is combined with a selector, then we create a wrapper function around the method after applying it to the object. $\mathit{MkCall}$ values are then used by \emph{e.g.} the $\mathit{object}$ and $\mathit{result}$ combinators as follows:
\begin{displaymath}
\begin{array}{l}
\mathit{object} :: \mathit{Selector}~o~s~\mathit{Identity}~\mathit{arg}~\mathit{ret} \to arg \to o\\
\mathit{object}~(\mathit{MkCall}~\mathit{call})  = \mathit{snd} \circ \mathit{call}
\end{array}
\end{displaymath}

