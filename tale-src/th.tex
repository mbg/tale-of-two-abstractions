\section{Template Haskell}
\label{sec:th}

We have developed a Haskell library\footnote{\url{https://github.com/mbg/monadic-state-hierarchies} on GitHub and \url{http://hackage.haskell.org/package/msh} on Hackage} which implements the automatic translation from a simple object-oriented language to the encoding shown in Section \ref{sec:auto}. Object classes can be defined inside of a Haskell source file with the help of a Template Haskell $\mathit{QuasiQuoter}$\cite{mainland2007s}. The example for a simple expression language we have shown in Section \ref{sec:encoding} using Java syntax can easily be converted into our state declarations:
\begin{displaymath}
\begin{array}{l}
[\mathit{state}\mid\\\mathbf{abstract}~\mathbf{state}~\mathit{Expr}~\mathbf{where} \\
\quad \begin{array}{lcl}
\mathit{eval} & :: & \mathit{Int}\\
\end{array}\\\\
\mathbf{state}~\mathit{Add} : \mathit{Expr}~\mathbf{where} \\
\quad \begin{array}{lcl}
\mathbf{data}~\mathit{left} & :: & \mathit{Expr} \\
\mathbf{data}~\mathit{right}  & :: & \mathit{Expr}
\end{array}\\\\
\quad \begin{array}{lcl}
\mathit{eval} & :: & \mathit{Int}\\
\mathit{eval} & = & \mathbf{do} \\
\multicolumn{3}{l}{\quad  \begin{array}{lcl}
x & \leftarrow & \mathit{this}.!\mathit{left}.!\mathit{eval}\\
y & \leftarrow & \mathit{this}.!\mathit{right}.!\mathit{eval}\\
\multicolumn{3}{l}{\mathit{return}~(x + y) }
\end{array}} 
\end{array}\\
\mid]
\end{array}
\end{displaymath}
The quotation is parsed and then translated into an abstract representation of Haskell's syntax, according to the rules shown in Section \ref{sec:auto}. Method typings must be given, similar to type classes. We parse them as standard Haskell, but their co-domains will be wrapped into \emph{e.g.} $\mathit{AddM}$ in the example above. Method equations are also parsed as normal Haskell, but no transformations are applied. The code for this and other examples may be found in our code repository.

\subsection{Selectors}

In order to allow programmers to work with the object classes, our library contains a number of combinators. Most of these combinators operate on \emph{selectors} which are defined using a GADT and are indexed over whether they represent a mutable member, \emph{i.e.} a field, or an immutable member, \emph{i.e.} a method. The $\mathit{MemberType}$ type is promoted to the kind level and the $\mathit{Mutable}$ and $\mathit{Immutable}$ constructors are used as types:
\begin{displaymath}
\begin{array}{l}
\mathbf{data}~\mathit{MemberType} = \mathit{Mutable} \mid \mathit{Immutable} \\\\
\mathbf{data}~\mathit{Selector}~(\mathit{ty} :: \mathit{MemberType})~o~s~m~a~\mathbf{where} \\
\quad \begin{array}{lcl}
 \mathit{MkMethod} & :: & \mathit{StateT}~s~m~a \to \\
                   &    & (o \to m~(a, o)) \to\\
                   &    & \mathit{Selector}~\mathit{Immutable}~o~s~m~a \\
\mathit{MkField}   & :: & (o \to m (a,o)) \to \\
                   &    & \mathit{StateT}~s~m~a \to \\
                   &    & (o \to a \to m~((),o)) \to \\
                   &    & (a \to \mathit{StateT}~s~m~()) \to \\
                   &    & \mathit{Selector}~\mathit{Mutable}~o~s~m~a
\end{array}
\end{array}
\end{displaymath}
Intuitively, a selector is a value which contains both, internal and external, versions of a member so that the same name can be used for both. For fields, it contains both versions for the getter and the setter. 
The type classes which are generated for object classes contain selectors for all of their members, using the original names from the state declaration. For example, including inherited members and in addition to the type class methods shown in Section \ref{sec:auto}, the $\mathit{Add}$ object class contains:
\begin{displaymath}
\begin{array}{l}
\mathit{left} :: \mathit{AddLike}~o~s~m \Rightarrow \\
\qquad \mathit{Selector}~\mathit{Mutable}~o~s~m~\mathit{Expr} \\
\mathit{right} :: \mathit{AddLike}~o~s~m \Rightarrow \\
\qquad \mathit{Selector}~\mathit{Mutable}~o~s~m~\mathit{Expr} \\
\mathit{eval} :: \mathit{EvalLike}~o~s~m \Rightarrow \\
\qquad \mathit{Selector}~\mathit{Immutable}~o~s~m~\mathit{Int} \\
\end{array}
\end{displaymath}
Selectors allow us to use member names directly in either an internal or external call context without any compile-time transformations which resolve them to the appropriate functions. For example, in the type class instance for $\mathit{Add}$, $\mathit{left}$ has the following value:
\begin{displaymath}
\mathit{left} = \mathit{MkField}~\mathit{\_ext\_get\_next}~\mathit{\_int\_get\_next}~\mathit{\_ext\_set\_next}~\mathit{\_int\_set\_next}
\end{displaymath}

\subsection{Selector composition}

In object-oriented languages, member access can be chained together using the $.$ ``operator''. For example, $\mathit{obj}.\mathit{foo}.\mathit{bar}$ gets the value of a field $\mathit{foo}$ belonging to an object $\mathit{obj}$ and then invokes the $\mathit{bar}$ method on the value of $\mathit{foo}$. We have a similar mechanism in our library. However, since $.$ is already used for function composition, we use $.!$ instead. This operator is part of a type class $\mathit{Object}$, which is added as a super-class constraint to every type class generated for the interface of a base class:
\begin{displaymath}
\begin{array}{l}
\mathbf{infixr}~8~.!\\
\mathbf{class}~\mathit{Monad}~m \Rightarrow \mathit{Object}~\mathit{obj}~\mathit{st}~m~\mathit{where} \\
\qquad \begin{array}{lcl}
\mathit{this} & :: & \mathit{This}~\mathit{obj}~\mathit{st}~\mathit{m}~\mathit{obj} \\
\mathit{this} & = & \mathit{MkThis} \\\\

(.!) & :: & \forall r~\mathit{ty}.\\ 
     &    & \mathit{obj} \to \\
     &    & \mathit{Selector}~\mathit{ty}~(\mathit{QueryObject}~\mathit{obj})~\mathit{st}~(\mathit{QueryMonad}~\mathit{obj}~\mathit{m})~r \to  \\
     &    & \mathit{QueryResult}~\mathit{obj}~\mathit{ty}~\mathit{st}~m~r
\end{array}
\end{array}
\end{displaymath}
Note that the $\mathit{This}$ type is defined as $\mathbf{data}~\mathit{This}~o~s~m~a = \mathit{MkThis}$ and simply serves to capture the type arguments of instances of the $\mathit{Object}$ type class. The type of $.!$ is highly flexible. The left argument of the operator can be any type that is made an instance of the $\mathit{Object}$ class. Primarily, those are objects as the name would imply, but selectors and $\mathit{This}$ are also instances of $\mathit{Object}$. The right argument is always a selector. We use the $\mathit{QueryObject}$ and $\mathit{QueryMonad}$ type families to determine what object the selector will be accessing and what monad stack it will run on, depending on the left argument of $.!$.

The $.!$ operator is right-associative so that we can build up larger selectors and ultimately decide whether the call should be internal or external, depending on whether we apply it to an object or $\mathit{this}$.

The type of selector we obtain from composing two selectors is determined by a closed type function:
\begin{displaymath}
\begin{array}{l}
\mathbf{type}~\mathbf{family}~\mathit{MemberComposeResult} \\
\quad (\mathit{lhs} :: \mathit{MemberType})~(\mathit{rhs} :: \mathit{MemberType} :: \mathit{MemberType})~\mathbf{where} \\\\
\qquad \begin{array}{lcl}
\mathit{MemberComposeResult}~\mathit{Immutable}~\mathit{Immutable} & = & \mathit{Immutable} \\
\mathit{MemberComposeResult}~\mathit{Immutable}~\mathit{Mutable} & = & \mathit{Immutable} \\
\mathit{MemberComposeResult}~\mathit{Mutable}~\mathit{Immutable} & = & \mathit{Immutable} \\
\mathit{MemberComposeResult}~\mathit{Mutable}~\mathit{Mutable} & = & \mathit{Mutable} 
\end{array}
\end{array}
\end{displaymath}
Intuitively, the result of composing two selectors can only be a mutable selector if both of the original selectors are also mutable. 

In most object-oriented languages, $\mathit{null}$ is automatically a value of all reference types. Members of a field or variable can be accessed without checking whether the value is $\mathit{null}$ or not, frequently leading to runtime errors which are difficult to track down. Functional languages, such as ML or Haskell, do not have $\mathit{null}$ values -- they have a better solution: the option type. In Haskell, this type is called $\mathit{Maybe}$ and is defined as:
\begin{displaymath}
\begin{array}{lcl}
\mathbf{data}~\mathit{Maybe}~a & = & \mathit{Nothing} \mid \mathit{Just}~a
\end{array}
\end{displaymath}
In other words, a value of type $\mathit{Maybe}~a$ is either $\mathit{Nothing}$ -- the rough equivalent of $\mathit{null}$ -- or $\mathit{Just}~x$ for some $x :: a$. Case analysis must be performed on every such value to determine whether it contains a value of type $a$ or not. Functions expecting an argument of type $a$ cannot be applied to a value of type $\mathit{Maybe}~a$. Object-oriented languages have recently begun to adapt this type as a type-safe alternative in favour of $\mathit{null}$. However, checking whether a value is $\mathit{Nothing}$ or not can still be tedious if, for example, we wish to only run some function if a certain value is not $\mathit{null}$. C\#, for example, has recently introduced the $.?$ operator which can be used in place of $.$. The member on the right of $.?$ will only be accessed if the value on the left is not the equivalent of $\mathit{Nothing}$.

As functional programmers, we know that C\#'s $.?$ operator is really just a specialised version of $\mathit{fmap}$, because the option type is a functor. In our library, we have a $.\$$ operator in addition to $.!$, which fulfils the same role as C\#'s $.?$ operator, except that it is not specialised to the option time. Instead, it works for any type that is an instance of the $\mathit{Functor}$ type class. We believe that this is further evidence that, by combining ideas from both paradigms, we can 

\subsection{Combinators}

Field selectors can be assigned new values in the definition of a method using the $<:$ operator. It is defined as follows: 
\begin{displaymath}
\begin{array}{l}
(<:)  ::  \mathit{Monad}~m \Rightarrow \mathit{Selector}~\mathit{Mutable}~o~s~m~\mathit{val} \to \mathit{val} \to \mathit{StateT}~s~m~()\\
(\mathit{MkField}~\_~\_~\_~s') <: val  =  s' val
\end{array}
\end{displaymath}
Note that method selectors cannot be supplied as an argument to $<:$ since we explicitly require the selector to be tagged as $\mathit{Mutable}$. Haskell's type system then ensures that this is adhered to.

Every external call returns a pair consisting of a result as well as an updated object. A language like Java would just mutate the object on which the external call was made, but since our system is pure and values are immutable, this cannot be done. Surrounding external calls with $\mathit{fst}$ and $\mathit{snd}$ may seem a bit cryptic, so instead we provide aliases named $\mathit{result}$ and $\mathit{object}$ for the projections, respectively. 

We will often want to perform case analysis on the value of a member within a method. Instead of first binding the value to a variable and then using $\mathbf{case}$ on it, we provide a shortcut. The $\mathit{switch}$ combinator has type $\mathit{Monad}~m \Rightarrow \mathit{Selector}~\mathit{ty}~o~s~m~\mathit{val} \to (\mathit{val} \to \mathit{StateT}~s~m~b) \to \mathit{StateT}~s~m~b$.

As an example of these combinators, consider an implementation of linked lists in the object-oriented style:

\begin{displaymath}
\begin{array}{l}
\mathbf{state}~\mathit{ListItem}~a~\mathbf{where} \\
\qquad \begin{array}{lcl}
\mathbf{data}~\mathit{value} & :: & a \\
\mathbf{data}~\mathit{next}  & :: & \mathit{Maybe}~(\mathit{ListItem}~a)
\end{array}\\\\
\qquad \begin{array}{lcl}
\mathit{appendEnd} & :: & \mathit{ListItem}~a \to () \\
\mathit{appendEnd}~\mathit{item} & = & \mathbf{do} \\
 \multicolumn{3}{l}{\qquad \begin{array}{l}
\mathit{switch}~\mathit{next}~\$~\lambda case~x \to \\
\qquad \begin{array}{lcl}
\mathit{Nothing} & \to & \mathit{next} <: \mathit{Just}~\mathit{item} \\
\mathit{(Just~\_)} & \to & \mathit{this}.!\mathit{next}.\$\mathit{appendEnd}~\mathit{item}
\end{array}
\end{array}}
\end{array}
\end{array}
\end{displaymath}

\subsection{Normal type classes}

Objects generated by our library can be made instances of ``normal'' Haskell type classes, removing the need for object-oriented-style interfaces. For example, we could write the following to make $\mathit{Val}$ objects an instance of the $\mathit{Show}$ type class:
\begin{displaymath}
\begin{array}{l}
\mathbf{instance}~\mathit{Show}~\mathit{Val}~\mathbf{where}\\
\qquad \begin{array}{lcl}
\mathit{show}~\mathit{obj} & = & \mathit{show}~(\mathit{result}~(\mathit{obj}.!\mathit{value})) 
\end{array}
\end{array}
\end{displaymath}
We believe that this, together with the flexibility of selectors, shows that our system is not an entirely different world from existing Haskell code, but can be used in combination with it.