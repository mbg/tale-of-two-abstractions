\section{Subtyping}
\label{sec:encoding}

A pleasant property of class hierarchies in object-oriented languages is that they are open for extension.

\textbf{COPIED FROM OLD VERSION}

Our encoding is based on the technique of combining existential types with type classes to implement dynamic dispatch, as described in Section \ref{sec:introduction}. We observe that this basic technique corresponds to an interface in the object-oriented paradigm. It has no state and does not implement any methods, but merely serves as a base type for types which implement the methods described by the type class constraint. Values of such types can be ``cast'' to the base type by wrapping them inside the constructor. Therefore, a notation of subtyping exists on the value level as well as on the type level.

We now show how to expand this technique to a full object system supporting all of the key features of the object-oriented paradigm, with the exception of aliasing. Beginning with encapsulation, we define a single-constructor data type to store the data of each object class. For example, for the $\mathit{MList}$ class shown in Section \ref{sec:usage}, we define the following\footnote{For convention, we use a prefix underscore to indicate that a function is only used internally by the encoding.}:
\begin{displaymath}
\begin{array}{lcl}
\mathbf{data}~\mathit{MListState}~a & = & \mathit{MkMListState}~\{\\
\multicolumn{3}{l}{\quad \_ \mathit{list} \_ \mathit{root} :: \mathit{Maybe}~(\mathit{MListItem~a}) } \\
\} &&
\end{array}
\end{displaymath}
This data type is used as parameter for a state monad transformer from Haskell's \texttt{mtl} library\footnote{\url{http://hackage.haskell.org/package/mtl}}. If a class has no parent, as is the case with $\mathit{MList}$, then it is a transformer over the $\mathit{Identity}$ monad. If a class has a parent, such as $\mathit{SList}$, then the $\mathit{Identity}$ monad is replaced with the state monad transformer corresponding to its parent class:
\begin{displaymath}
\begin{array}{lcl}
\mathbf{type}~\mathit{MListM}~a & = & \mathit{StateT}~(\mathit{MListState}~a)~\mathit{Identity} \\
\mathbf{type}~\mathit{SListM}~a & = & \mathit{StateT}~(\mathit{SListState}~a)~(\mathit{MListM}~a)
\end{array}
\end{displaymath}
State transformers\footnote{Not to be confused with state monad transformers} are functions $s \to (a,s)$ which accept an initial state of type $s$ as argument and return a result of some type $a$ as well as an updated state. The state monad captures this effect and allows state transformers to be ordered so that the updated state of one is given to the next as initial state. The monad transformer version of the state monad, known as state monad transformer, allows state monads to be layered on top of other monads. The resulting monad stacks have all of their monads' effects layered on top of each other. Therefore, a stack of state monad transformers will propagate the state of each layer. Or in other words, the top-most state monad transformer inherits the state of all other state monads in the stack.

Let us note that $\mathit{StateT}~s~(\mathit{StateT}~t~\mathit{Identity})$ is isomorphic to $\mathit{StateT}~(s,t)~\mathit{Identity}$ and either representation could be chosen. However, we prefer the former for simplicity since it allows us to construct the state one layer at a time. This is necessary due to the representation of objects in our system and is discussed in more detail later on in this section. We want to parametrise each state monad transformer with only the state of the corresponding object. Therefore, data types representing the state of sub-classes do not include any fields inherited from their parent(s):
\begin{displaymath}
\begin{array}{lcl}
\mathbf{data}~\mathit{SListState}~a & = & \mathit{MkSListState}~\{ \\
\multicolumn{3}{l}{\quad \_ slist \_ pred :: a \to a \to \mathit{Bool}}\\
\} &&
\end{array}
\end{displaymath}
We use a multi-parameter type class together with functional dependencies \cite{jones2000type} to represent the interface of an object class. We can further divide such an interface into two groups of methods. Each method of an object class has a corresponding function in both groups. While all are part of the same type class, one group is used internally by the encoding when the call-site is a method within the same object class. The other group of methods can be called on objects from all other call-sites, such as \emph{e.g.} pure code. They are responsible for unfolding an object into a stack of state monad transformers, invoking the corresponding internal method on the stack, and then folding the updated state back into an object. For example, the following is a partial definition of the type class for $\mathit{MList}$ showing the two methods corresponding to the $\mathit{insert}$ method in the object class:
\begin{displaymath}
\begin{array}{l}
\mathbf{class}~\mathit{Monad}~m \Rightarrow \mathit{MListLike}~o~s~m~a \mid o \to s, s \to o~\mathbf{where} \\
\quad \begin{array}{lcl}
\_\mathit{get}\_\mathit{list}\_\mathit{root} & :: & o~a \to m~(\mathit{Maybe}~(\mathit{MItem~a}), o~a)\\
\_\mathit{get}\_\mathit{list}\_\mathit{root}' & :: & \mathit{StateT}~(s~a)~m~(\mathit{Maybe}~(\mathit{MItem}~a))\\
\_ecall\_\mathit{insert} & :: & o~a \to a \to m~((),o~a) \\
\_icall\_\mathit{insert} & :: & a \to \mathit{StateT}~(s~a)~m~()
\end{array}
\end{array}
\end{displaymath}
Each type class belonging to an object class has at least three parameters: $o$ is the type representing objects, $s$ is the data type describing the state, and $m$ is the monad in which methods of this type class can be invoked in. Additionally, there is one type parameter for each type parameter of the object class, which is just $a$ for $\mathit{MList}$. The two functional dependencies $o \to s$ and $s \to o$ say that the object and state types are uniquely determined by each other. In other words, a type class instance for $\mathit{MListLike}$ can be selected by Haskell's type inference based solely on a known $o$ or a known $s$.

We combine all of the above to define a data type for objects belonging to an object class. The value representation of an object is reminiscent of a zipper \cite{huet1997zipper}. This data structure is used for the bi-directional traversal of another data structure, such as a list or a tree, and consists of three parts. For \emph{e.g.} a functional list the zipper consists of a list of ancestor elements which have already been traversed, the element which is currently in view, and a list of successor elements which have not been traversed yet. Our encoding of an object is essentially a zipper over a heterogeneous list of partial objects which we will refer to as $\Delta$-objects. A $\Delta$-object contains only the state of the corresponding class.

When an object is constructed, the view of the zipper is placed on the last element in the list -- the $\Delta$-object corresponding to the object class which is being instantiated. Its immediate ancestor is a $\Delta$-object corresponding to its superclass, which in turn may have its superclass as an ancestor, and so on. Such an object can be cast to a superclass by shifting the zipper's view to an ancestor in which case the sub-class becomes a successor element (Figure \ref{fig:cast}). In this case, the type of the $\Delta$-object representing the sub-class is existentially-quantified and we no longer know anything about it, other than that it implements the same type class as the parent. For this reason, the zipper must first be traversed to construct the state of the whole object. it is therefore beneficial to be able to do this in layers using multiple state monad transformers.

\tikzstyle{int}=[draw, minimum size=2em]
\tikzstyle{init} = [pin edge={thin,gray}]

\begin{figure}
    \begin{center}
        \bgroup
        \def\arraystretch{1.5}
        \begin{tabular}{ccc}
            \begin{tikzpicture}[node distance=2.0cm,auto,>=latex']
            \node [int] (a) {$\mathit{MList}$};
            \node [int,below=1cm,pin={[init]above:View}] (b) [right of=a] {$\mathit{SList}$};
            \path[->] (b) edge node {} (a);
            \end{tikzpicture} & $\qquad \Leftrightarrow \qquad$ &
            
            \begin{tikzpicture}[node distance=2.0cm,auto,>=latex']
            \node [int,pin={[init]above:View}] (c)  {$\mathit{MList}$};
            \node [int,above=1cm] (d) [right of=c] {$\mathit{SList}$};
            \path[->] (c) edge node {} (d);
            \end{tikzpicture} \\
            $\mathit{SList}$ object & & $\mathit{MList}$ object
        \end{tabular}
        \egroup
    \end{center}
    \caption{Cast between an $\mathit{SList}$ object and an $\mathit{MList}$ object} \label{fig:cast}
\end{figure}

Each object class has its own data type used to represent objects. These are tailored to account only for the zipper states that objects of a class can actually be in. For example, the encoding of $\mathit{MList}$ objects is shown below: 
\begin{displaymath}
\begin{array}{lcl}
\mathbf{data}~\mathit{MList}~a & = & \mathit{MListData}~\{ \\ 
\multicolumn{3}{l}{\quad \_\mathit{list}\_\mathit{data} :: \mathit{MListState}~a}\\
\multicolumn{3}{l}{\begin{array}{lcl}
    \} & \mid & \forall s~d.\mathit{MListLike}~s~d~(\mathit{MListM}~a)~a \Rightarrow \\
    & &  \mathit{MListStart}~\{
    \end{array}}  \\
\multicolumn{3}{l}{\begin{array}{lcl}
    \_\mathit{list}\_\mathit{data} & :: & \mathit{MListState}~a,\\
    \_\mathit{list}\_\mathit{sub}  & :: & s~a
    \end{array}}\\
\} &&
\end{array}
\end{displaymath}
An $\mathit{MList}$ object has two possible states: either it is an instance of $\mathit{MList}$ or it is an instance of a sub-class of $\mathit{MList}$ which has been cast to $\mathit{MList}$. The $\mathit{MListData}$ and $\mathit{MListStart}$ constructors represent these cases respectively if this $\Delta$-object is in the view. If it is not, then the $\mathit{MListData}$ constructor may also be used to represent an ancestor to another $\Delta$-object. Figure \ref{tab:baseconstructors} shows how class attributes affect which constructors are required.
\begin{figure}
    \begin{center}
        \begin{tabular}{|c|c|c|c|}
            \hline \emph{Base class} & None & Abstract & Final \\ 
            \hline $\mathit{Data}$         & x & x & x \\ 
            \hline $\mathit{Start}$     & x & x & \\
            \hline
        \end{tabular} 
        \begin{tabular}{|c|c|c|c|}
            \hline \emph{Sub-class} & None & Abstract & Final \\ 
            \hline $\mathit{Data}$         & x &   & x \\ 
            \hline $\mathit{Start}$     & x & x &   \\ 
            \hline $\mathit{End}$     & x &   & x \\ 
            \hline $\mathit{Middle}$ & x & x &   \\ 
            \hline 
        \end{tabular} 
    \end{center}
    \caption{Object constructors by attribute}
    \label{tab:baseconstructors}
\end{figure}

Subclasses have more possible states because, unlike base class objects, they may have ancestors. For illustration, the encoding of $\mathit{SList}$ objects is given below:
\begin{displaymath}
\begin{array}{lcl}
\mathbf{data}~\mathit{SList}~a & = & \mathit{SListData}~\{ \\ 
\multicolumn{3}{l}{\quad \_\mathit{list}\_\mathit{data} :: \mathit{SListState}~a}\\
\multicolumn{3}{l}{\begin{array}{lcl}
    \} & \mid & \forall s~d.\mathit{SListLike}~s~d~(\mathit{SListM}~a)~a \Rightarrow \\
    & &  \mathit{SListStart}~\{
    \end{array}}  \\
\multicolumn{3}{l}{\quad \begin{array}{lcl}
    \_\mathit{list}\_\mathit{data} & :: & \mathit{SListState}~a,\\
    \_\mathit{list}\_\mathit{sub}  & :: & s~a
    \end{array}}\\
\multicolumn{3}{l}{ \begin{array}{lcl}
    \} & \mid & \mathit{SListEnd}~\{ 
    \end{array}}  \\
\multicolumn{3}{l}{\quad \begin{array}{lcl}
    \_\mathit{slist}\_\mathit{sup}  & :: & \mathit{MList}~a, \\
    \_\mathit{slist}\_\mathit{data} & :: & \mathit{SListState}~a
    \end{array}}\\
\multicolumn{3}{l}{ \begin{array}{lcl}
    \} & \mid & \forall s~d.\mathit{SListLike}~s~d~(\mathit{SListM}~a)~a \Rightarrow \\
    & &  \mathit{SListMiddle}~\{
    \end{array}}  \\
\multicolumn{3}{l}{ \quad \begin{array}{lcl}
    \_\mathit{slist}\_\mathit{sup}  & :: & \mathit{MList}~a, \\
    \_\mathit{slist}\_\mathit{data} & :: & \mathit{SListState}~a, \\
    \_\mathit{slist}\_\mathit{sub}  & :: & s~a
    \end{array}}\\
\} &&
\end{array}
\end{displaymath}

$\mathit{SList}$ is a sub-class that is neither abstract nor final. However, like $\mathit{MList}$, it can only be in one of two possible states while it is in view. If it is an instance of $\mathit{SList}$, then it will have an ancestor no successors which is represented by the $\mathit{SListEnd}$ constructor. If it is an instance of a sub-class of $\mathit{SList}$, then it does additionally have successors and the $\mathit{SListMiddle}$ constructor is used. Depending on whether a $\Delta$-object of this type is an instance of $\mathit{SList}$ or a sub-class, $\mathit{SListData}$ or $\mathit{SListStart}$ are used to represent it as a successor, respectively.

Instantiating new objects requires us to set up the initial zipper configuration. This is implemented in the $\mathit{new}$ function, which is a member of a type class so it can be overloaded for every type of object (Section \ref{sec:th}). For example, to do this for $\mathit{SList}$ objects we have:
\begin{displaymath}
\begin{array}{lcl}
\mathit{new} & :: & (\mathit{Maybe}~(\mathit{MListItem}~a), (a \to a \to \mathit{Bool})) \to \\
&&\mathit{SList}~a \\
\mathit{new}~r~p & = & \mathit{SListEnd}~\{ \\
\multicolumn{3}{l}{\quad \begin{array}{lcl}
    \_\mathit{slist}\_\mathit{sup} & = & \mathit{new}~r, \\
    \_\mathit{slist}\_\mathit{data} & = & \mathit{MkSListState}~\{\\
    \multicolumn{3}{l}{\quad \_\mathit{slist}\_\mathit{pred} = p}\\ 
    \end{array}} \\
\}\} &&
\end{array}
\end{displaymath}
In order to cast from \emph{e.g.} an $\mathit{SList}$ object to an $\mathit{MList}$ object, the view of the zipper must be shifted as shown previously in Figure \ref{fig:cast}. The implementation of the illustrated cast is given below. This $\mathit{cast}$ function is really part of a type class for which there are instances for every possible cast.
\begin{displaymath}
\begin{array}{lcl}
\multicolumn{3}{l}{\mathit{cast} :: \mathit{SList}~a \to \mathit{MList}~a} \\
\mathit{cast}~(\mathit{SListEnd}~(\mathit{MListData}~\mathit{pd})~\mathit{d}) & = & \\ \multicolumn{3}{l}{\quad \mathit{MListStart}~\mathit{pd}~(\mathit{SListData}~d)} \\
\mathit{cast}~(\mathit{SListMiddle}~(\mathit{MListData}~\mathit{pd})~\mathit{d}~\mathit{sub}) & = & \\ \multicolumn{3}{l}{\quad \mathit{MListStart}~\mathit{pd}~(\mathit{SListStart}~d~\mathit{sub})}
\end{array}
\end{displaymath}
Casts in the other direction require a bit more work. Firstly, we need a different casting function whose result is wrapped in $\mathit{Maybe}$ to indicate that a cast from a base-class to a sub-class may not succeed. However, due to the use of existential types, we also require runtime type information about the type of the value in \emph{e.g.} the $\_\mathit{list}\_\mathit{sub}$ field. Haskell's $\mathit{Typeable}$ type class\footnote{\url{https://hackage.haskell.org/package/base/docs/Data-Typeable.html}} may be used for this purpose.

Object types must also be made instances of their corresponding type class, as well as of all type classes belonging to their superclases. For example, part of the $\mathit{MListLike}$ instance for $\mathit{MList}$ is shown below:
\begin{displaymath}
\begin{array}{l}
\mathbf{instance}~\mathit{MListLike}~\mathit{MList}~\mathit{MListState}~\mathit{Identity}~a~\mathbf{where} \\
\quad \begin{array}{lcl}
\mathit{insert}~(\mathit{MListData}~d)~v & = & \mathbf{do} \\
\multicolumn{3}{l}{\quad \begin{array}{lcl}
    (r,d') & \leftarrow & \mathit{runStateT}~(\_\mathit{insert}~v)~d \\
    \multicolumn{3}{l}{\mathit{return}~(r,\mathit{MListData}~d')}
    \end{array}} \\
\mathit{insert}~(\mathit{MListStart}~d~s)~v & = & \mathbf{do} \\
\multicolumn{3}{l}{\quad \begin{array}{lcl}
    ((r,s'),d') & \leftarrow & \mathit{runStateT}~(\mathit{insert}~s~v)~d \\
    \multicolumn{3}{l}{\mathit{return}~(r,\mathit{MListStart}~d'~s')}
    \end{array}} \\
\_\mathit{insert} & = & \mathit{list}\_\mathit{insert}
\end{array}
\end{array}
\end{displaymath}
The $\mathit{insert}$ method implements part of the dynamic dispatch mechanism. It inspects the state of the object it is called on and calls $\mathit{MList}$'s implementation of $\mathit{insert}$ if it is an instance of $\mathit{MList}$ by setting up a state monad transformer with the object's current state and invoking $\_\mathit{insert}$ in it, before returning an updated object with the resulting state. If if it an instance of some sub-class, then $\mathit{insert}$ still sets up the state monad transformer, but calls the sub-classes' implementation of $\mathit{insert}$ instead. Unless overriden, getters and setters which are inherited from another class are not executed in the child's monad stack since they would be incompatible with it, but are instead $\mathit{lift}$ed to the parent's. This hides the boilerplate normally required for monad transformers.

For example, $\mathit{SList}$ inherits the $\mathit{root}$ field from $\mathit{MList}$. The $\mathit{MListLike}$ instance for $\mathit{MList}$ will have a definition:
\begin{displaymath}
\begin{array}{lcl}
\mathit{\_get\_list\_root'} & = & \mathit{gets}~\mathit{\_list\_root}
\end{array}
\end{displaymath}
Meanwhile, in the $\mathit{MListLike}$ instance for $\mathit{SList}$, we have:
\begin{displaymath}
\begin{array}{lcl}
\mathit{\_get\_list\_root'} & = & \mathit{lift}~\mathit{\_get\_list\_root'}
\end{array}
\end{displaymath}
Calling a method on an object sets up a stack of state monad transformers parametrised over each $\Delta$-object's state. If the object has the type of a base class, this stack can be set up in order by the dynamic dispatch mechanism. If an object has the type of a sub-class, its value representation will reflect this by having the $\Delta$-object corresponding to the sub-class in view. However, all methods of a sub-class require an appropriate monad stack to be set-up before they can be called and cannot be called directly from elsewhere.

Figure \ref{fig:subinvoke} shows the process by which our encoding removes this limitation. When a method is invoked on a sub-class object, the call gets forwarded to along the ancestors of the zipper to until it reaches the base class using a function named $C\_\mathit{invoke}$ for a class named $C$. The definition of this function is given later in this section. Once the base class object has been reached, it can then construct the monad stack from the bottom up as if a method was called on the base class object itself.

\begin{figure}
    \begin{center}
        \begin{tikzpicture}[node distance=5.0cm,auto,>=latex']
        \node [int] (c)  {$\mathit{MList}$};
        \node [int,pin={[init]above:View}] (d) [right of=c] {$\mathit{SList}$};
        \node (b) [below of=d,node distance=1cm, coordinate] {a};
        
        \path[->] ([yshift=1ex]c.east) edge node {$\mathit{runStateT~insert}$} ([yshift=1ex]d.west);
        \path[->] ([yshift=-1ex]d.west) edge node {$\mathit{list}\_\mathit{invoke}~\mathit{insert}$} ([yshift=-1ex]c.east);
        \path[->] (b.north) edge node [right] {$\mathit{obj.insert}$} (d.south);
        \end{tikzpicture} 
    \end{center}
    \caption{Invoking $\mathit{insert}$ on an $\mathit{SList}$ object $\mathit{obj}$} \label{fig:subinvoke}
\end{figure}

This process requires a second instance of the type class belonging to a sub-class where the underlying monad is the $\mathit{Identity}$ monad as opposed to the parent class's state monad. Methods in such an instance call their parents $\mathit{invoke}$ function with themselves as argument. For example, the type and implementation of $\mathit{MList}$'s $\mathit{invoke}$ function is shown below:
\begin{displaymath}
\begin{array}{l}
\mathbf{class}~\mathit{Monad}~m \Rightarrow \mathit{MListLike}~o~s~m~a \mid o \to s, s \to o~\mathbf{where} \\
\quad \begin{array}{lcl}
\_\mathit{list}\_\mathit{invoke} & :: & \mathit{MListLike}~o'~d'~(\mathit{StateT}~(s~a)~m)~a \Rightarrow\\
\multicolumn{3}{l}{\quad o'~a \to (o'~a \to \mathit{StateT}~(s~a)~m~(r,o'~a)) \to} \\
\multicolumn{3}{l}{\quad o~a \to m~(r,o~a,o'~a)}
\end{array} \\\\
\mathbf{instance}~\mathit{MListLike}~\mathit{MList}~\mathit{MListState}~\mathit{Identity}~a~\mathbf{where} \\
\quad \begin{array}{lcl}
\_\mathit{list}\_\mathit{invoke}~s~f~(\mathit{MListData}~d) & = & \mathbf{do} \\
\multicolumn{3}{l}{\quad \begin{array}{lcl}
    ((r,s'),d') & \leftarrow & \mathit{runStateT}~(f~s)~d \\
    \multicolumn{3}{l}{\mathit{return}~(r,\mathit{MListData}~d',s')}
    \end{array}}
\end{array}
\end{array}
\end{displaymath}
The type class constraint on $\_\mathit{list}\_\mathit{invoke}$ ensures that the sub-class's implementation of $\mathit{MListLike}$ where the underlying monad is $\mathit{MList}$'s state monad is selected. This $\mathit{invoke}$ function is used as shown below:
\begin{displaymath}
\begin{array}{l}
\mathbf{instance}~\mathit{SListLike}~\mathit{SList}~\mathit{SListState}~\mathit{Identity}~a~\mathbf{where} \\
\quad \begin{array}{lcl}
\mathit{insert}~(\mathit{SListEnd}~\mathit{sup}~\mathit{d})~v & = & \mathbf{do} \\
\multicolumn{3}{l}{\quad \begin{array}{lcl}
    (r,\mathit{sup}',\mathit{SListEnd}~\_~d') & \leftarrow & \\
    \multicolumn{3}{l}{\quad \_\mathit{list}\_\mathit{invoke}~(\mathit{SListEnd}~\mathit{sup}~d)~\mathit{sup}} \\
    \multicolumn{3}{l}{\mathit{return}~(r,\mathit{SListEnd}~\mathit{sup}'~d')}
    \end{array}}
\end{array}
\end{array}
\end{displaymath}

\section{Translation}
\label{sec:auto}

Encoding object classes for our system by hand is both repetitive and error-prone. We propose syntactic sugar resembling familiar notations used by object-oriented languages to automatically generate the corresponding encodings. Object classes are modelled using $\mathbf{state}$ declarations. The following is an overview of the structure of such a declaration:
\begin{displaymath}
\begin{array}{l}
[\mathbf{abstract} \mid \mathbf{final}]~\mathbf{state}~\mathit{C}~\overline{\mathit{tyvar}}~[: P~\overline{\mathit{type}}]~\mathbf{where} \\
\quad \overline{\mathbf{data}~\mathit{var}~[= \mathit{expr}] :: \mathit{type}} \\
\quad \overline{[\mathbf{abstract}]~\mathit{var} :: \mathit{type}} \\
\quad \overline{\mathit{equation}}
\end{array}
\end{displaymath}
The $\mathbf{state}$ keyword is optionally preceded by either an $\mathbf{abstract}$ or $\mathbf{final}$ modifier, but cannot have both. If the $\mathbf{abstract}$ modifier is specified, the class cannot be instantiated directly, but it may include method typings marked as $\mathbf{abstract}$ which do not have an accompanying binding. Unless a class which inherits from an abstract class is abstract itself, it must implement all of its ancestors' methods which are marked as $\mathbf{abstract}$ and are therefore lacking an implementation. If a class is marked as $\mathbf{final}$, then no other class can inherit from it.

The name of an object class $C$ follows the conventions for type constructors\footnote{In Haskell, this means that the name must start with an upper-case character.} and is followed by zero or more type variables. The optional $: \mathit{P}~\overline{\mathit{type}}$ component of the header of the declaration is used to specify the class's parent. $P$ must be the name of a non-final object class. There must be as many $\mathit{type}$s as there are type parameters in $P$'s definition. Any type variables used here must first be declared as type parameters of $C$.

The body of an object class declaration following the $\mathbf{where}$ keyword consists of three types of definitions which may appear in any order. A line starting with the $\mathbf{data}$ keyword is used to declare a field and must consist of at least a name and a type, but may optionally have an expression describing its default value. If no default value is specified, then it is assumed to be $\bot$. The fields in the object class declaration are desugared into fields for the state date type: 
\begin{displaymath}
\begin{array}{lcl}
\mathbf{data}~\mathit{CState}~\overline{\mathit{tyvar}} & = & \mathit{MkCState}~\{~\overline{\_\mathit{var} :: \mathit{type}}~\}
\end{array}
\end{displaymath}
The names of this data type and its single constructor can be chosen arbitrarily, since they are never exposed directly to the user. However, for consistency, we have used the name of the object class followed by $\mathit{State}$ for the name of the data type and $\mathit{Mk}$, followed by the name of the data type, for the constructor. This data type is parametrised over the same type parameters as $C$. The type alias for the state monad transformer parametrised over $\mathit{CState}$ follows:
\begin{displaymath}
\mathbf{type}~\mathit{C}~\overline{\mathit{tyvar}} = \mathit{StateT}~(\mathit{CState}~\overline{\mathit{tyvar}})~[\mathit{Identity} \mid PM~\overline{\mathit{type}}]
\end{displaymath}
By convention, the type alias is named after the object class. As with the state data type before, its type parameters are the same as in the header of the declaration for $C$. If the class has a parent, its underlying monad is that corresponding to $P$ applied to the type argument specified in the header of $C$. Otherwise, it is $\mathit{Identity}$. The last new type that results from a declaration for an object class $C$ is the data type representing objects of type $C$:
\begin{displaymath}
\mathbf{data}~\mathit{C}~\overline{\mathit{tyvar}} = \mathit{constructors}
\end{displaymath}
We refer to Figure \ref{tab:baseconstructors} for guidance on which constructors are depending on whether $C$ is a sub-class or not and which attributes it has. The $\mathit{Middle}$ constructor combines all possible fields which may be required by an object's internal state:
\begin{displaymath}
\begin{array}{l}
\forall s~d.\mathit{CLike}~s~d~(\mathit{C}~\overline{\mathit{tyvar}})~\overline{\mathit{tyvar}} \Rightarrow \mathit{CMiddle}~\{\\
\quad \begin{array}{lcl}
\_\mathit{C}\_\mathit{sup}  & :: & \mathit{PObject}~\overline{\mathit{type}}, \\
\_\mathit{C}\_\mathit{data} & :: & \mathit{CState}~\overline{\mathit{tyvar}}, \\
\_\mathit{C}\_\mathit{sub}  & :: & s~\overline{\mathit{tyvar}}
\end{array}\\
\}
\end{array}
\end{displaymath}
The existentially-quantified type variables $s$ and $d$ as well as the type class constraint are only required if the $\mathit{sub}$ field is present. $s$ refers to the object type of a sub-class and $d$ refers to the corresponding state type. The constraint enforces that the sub-class must implement all methods of $C$. The $\mathit{sup}$ field is only present in sub-classes, but never base classes. Its value is a $\Delta$-object belonging to $P$. Finally, the $\mathit{data}$ field is always present and stores the $\Delta$-object's state.

Method typings and fields are used to construct the type class representing $C$'s interface: 
\begin{displaymath}
\begin{array}{l}
\mathbf{class}~[\mathit{Monad}~m \mid \mathit{PLike}~o~s~m~\overline{\mathit{type}}] \Rightarrow \\
\quad \mathit{CLike}~o~s~m~\overline{\mathit{tyvar}} \mid o \to s, s \to o~\mathbf{where} \\
\qquad \begin{array}{l}
\_\mathit{C}\_\mathit{invoke} :: \mathit{CLike}~o'~d'~(\mathit{StateT}~(s~\overline{\mathit{tyvar}})~m)~\overline{\mathit{tyvar}} \Rightarrow \\
\qquad o'~\overline{\mathit{tyvar}} \to \\ \qquad (o'~\overline{\mathit{tyvar}} \to \mathit{StateT}~(s~\overline{\mathit{tyvar}})~m~(r,o'~\overline{\mathit{tyvar}})) \to\\\qquad
o~\overline{\mathit{tyvar}} \to m~(r,o~\overline{\mathit{tyvar}},o'~\overline{\mathit{tyvar}}) \\
\overline{\_\mathit{get}\_\mathit{field} :: o~\overline{\mathit{tyvar}} \to m~(\mathit{type}, o~\overline{\mathit{tyvar}}) } \\
\overline{\_\mathit{get}\_\mathit{field}' :: \mathit{StateT}~(s~\overline{\mathit{tyvar}})~m~\mathit{type}} \\
\overline{\_\mathit{set}\_\mathit{field} :: o~\overline{\mathit{tyvar}} \to \mathit{type} \to m~((),o~\overline{\mathit{tyvar}})} \\
\overline{\_\mathit{set}\_\mathit{field}' :: \mathit{type} \to \mathit{StateT}~(s~\overline{\mathit{tyvar}})~m~()} \\
\overline{\mathit{\_ecall\_method} :: o~\overline{\mathit{tyvar}} \to \overline{\mathit{arg} \to}~m~(\mathit{result},o~\overline{\mathit{tyvar}})}\\
\overline{\_\mathit{icall\_method} :: \overline{\mathit{arg} \to}~ \mathit{StateT}~(s~\overline{\mathit{tyvar}})~m~\mathit{result} }
\end{array}
\end{array}
\end{displaymath}

For each field, there are two getters and two setters. One pair is used internally for calls within methods belonging to the same class, since the monad stack is already set up. The other is used when invoked on an object and sets up the monad stack before invoking the internal getter/setter. For every method, there are also two typings -- one used internally by methods in the same class and one used everywhere else. Additionally, the $\_\mathit{C}\_\mathit{invoke}$ method is used by sub-classes to invoke their own methods in a monad stack set up by $C$.

An instance of this type class is generated as well as one for each type class belonging to an ancestor of $C$. Each external method's implementation pattern matches on the configuration of the object, adds a state monad transformer layer on top of the existing monad stack, and then either invokes its own, internal implementation of the method or propagates the call to a sub-class.

If $C$ is not a base class, a second instance of $\mathit{CLike}$ is generated with the underlying monad set to $\mathit{Identity}$. In this case, all methods use $\_P\_\mathit{invoke}$ on the $\Delta$-object belonging to $P$ to perform the process shown in Figure \ref{fig:subinvoke}:
\begin{displaymath}
\begin{array}{l}
\mathbf{instance}~\mathit{CLike}~\mathit{C}~\mathit{CState}~\mathit{Identity}~\overline{\mathit{tyvar}}~\mathbf{where}\\
\quad \begin{array}{lcl}
\mathit{method}~(\mathit{CEnd}~\mathit{sup}~\mathit{d}) & = & \mathbf{do} \\
\multicolumn{3}{l}{\quad \begin{array}{lcl}
    (r,sup',\mathit{CEnd}~\_~d') & \leftarrow & \\ 
    \multicolumn{3}{l}{ \quad \_P\_\mathit{invoke}~(\mathit{CEnd}~\mathit{sup}~d)~\mathit{method}~\mathit{sup} }\\
    \multicolumn{3}{l}{\mathit{return}~(r, \mathit{CEnd}~sup'~d')}
    \end{array}}
\end{array}
\end{array}
\end{displaymath}
%Methods are mostly standard Haskell. $<:$ and $>:$ statements are desugared into calls to the appropriate getters or setters for a field. Method invocations (identifiers containing dots) are rewritten so that the object becomes the first argument of the method and the resulting expression is then wrapped into $\mathit{runIdentity}$.
