\section{Subtyping}
\label{sec:encoding}

A pleasant property of class hierarchies in object-oriented languages is that they are open for extension. In other words, classes do not need to know about other classes which inherit from them. For example, consider the following, abstract base class for an expression language in Java:
\begin{displaymath}
\begin{array}{l}
\mathbf{abstract}~\mathbf{class}~\mathit{Expr}~\{\\
\qquad \mathbf{public}~\mathbf{abstract}~\mathit{int}~\mathit{eval}();\\
\}
\end{array}
\end{displaymath}
This roughly corresponds to an algebraic data type $\mathit{Expr}$ in Haskell with no constructors and a function $\mathit{eval} :: \mathit{Expr} \to \mathit{Int}$. However, while we cannot retrospectively add constructors to a data type in Haskell, we can extend the $\mathit{Expr}$ class in Java:
\begin{displaymath}
\begin{array}{l}
\mathbf{class}~\mathit{Val}~\mathbf{extends}~\mathit{Expr}~\{\\
\qquad \mathbf{private}~\mathit{int}~\mathit{value};\\\\
\qquad \mathbf{public}~\mathit{int}~\mathit{eval}()~\{\\
\qquad \qquad \mathbf{return}~\mathit{this}.\mathit{value};\\
\qquad \}\\
\}
\end{array}
\end{displaymath}
The author of the $\mathit{Expr}$ class does not need to know about the $\mathit{Val}$ class and may even compile a library only containing $\mathit{Expr}$. The class can still be extended in other libraries by other authors, however. Our object system from Section \ref{sec:objects} allows for the same, but it has one major limitation: there is no support for subtyping that would allow us to \emph{e.g.} cast a $\mathit{DbModel}$ object to a $\mathit{MyDatabase}$ object. To see why this is an important limitation, consider the following sub-class for $\mathit{Expr}$ in Java:
\begin{displaymath}
\begin{array}{l}
\mathbf{class}~\mathit{Add}~\mathbf{extends}~\mathit{Expr}~\{\\
\qquad \mathbf{private}~\mathit{Expr}~\mathit{left};\\
\qquad \mathbf{private}~\mathit{Expr}~\mathit{right};\\\\
\qquad \mathbf{public}~\mathit{int}~\mathit{eval}()~\{\\
\qquad \qquad \mathbf{return}~\mathit{this}.\mathit{left}.\mathit{eval}() + \mathit{this}.\mathit{right}.\mathit{eval}();\\
\qquad \}\\
\}
\end{array}
\end{displaymath}
Without subtyping, the $\mathit{Add}$ class would be useless. There would be no possible values for the $\mathit{left}$ and $\mathit{right}$ fields, since $\mathit{Expr}$ itself is abstract and cannot be instantiated. It is only useful if sub-classes of $\mathit{Expr}$, which can be instantiated, can be used as $\mathit{Expr}$ values. In the remainder of this section, we show how to modify the encodings for our object system to support \emph{coercive subtyping} in Haskell.

\subsection{Objects as heterogenous zippers}

Objects so far have been represented using what are essentially singly-linked lists of $\Delta$-objects in which the $\Delta$-objects of sub-classes store their parent's $\Delta$-object. This in turn may contain its parent's $\Delta$-object, and so on. This is easy, because sub-classes always know the type of their parent. The inverse, however, is not true. Parental classes do not know what their children are. In order to represent a sub-class object as a value of its parent's type, the definition of the parent's data type must accommodate for all possible sub-classes, without knowing anything about them, except that they have at least the same interface. We can express this on the type-level using existential types. For example, for a base class we might have the following:
\begin{displaymath}
\begin{array}{lcl}
\mathbf{data}~\mathit{Parent} & = & \forall \mathit{sub}~\mathit{st}.\mathit{ParentCtx}~\mathit{sub}~\mathit{st}~(\mathit{StateT}~\mathit{ParentData}~\mathit{Identity}) \Rightarrow \\
&& \quad\mathit{MkParentStart}~\mathit{ParentData}~\mathit{sub}
\end{array}
\end{displaymath}
The $\mathit{sub}$ type variable, despite the $\forall$, is existentially-quantified. Together with the type class constraint, only types which implement at least the same interface as $\mathit{Parent}$ can be used as arguments to the $\mathit{ParentCtx}$ data constructor. Indeed, this technique is well-known in folklore for enabling dynamic dispatch, although it is considered an anti-pattern. However, we combine it with our previous representation for objects to encode coercive subtyping in Haskell. The resulting representation of objects is reminiscent of a zipper\cite{huet1997zipper}. This data structure is used for the bi-directional traversal of another data structure, such as a list or a tree, and consists of three parts. For \emph{e.g.} a functional list, its zipper consists of a list of ancestor elements which have already been traversed, the element which is currently in \emph{view}, and a list of successor elements which have not yet been traversed. Our encoding of an object is essentially a zipper over a heterogeneous list of $\Delta$-objects. 

Each object class still has its own data type used to represent objects. These are now tailored to account only for the zipper states that objects of a class can actually be in. For example, the encoding of $\mathit{MyDatabase}$ objects is shown below: 
\begin{displaymath}
\begin{array}{lcl}
\mathbf{data}~\mathit{MyDatabaseObj} & = & \mathit{MyDatabaseData}~\{ \\ 
\multicolumn{3}{l}{\quad \_\mathit{MyDatabase}\_\mathit{data} :: \mathit{SqlDatabase}}\\
\multicolumn{3}{l}{\begin{array}{lcl}
    \} & \mid & \forall s~d.\mathit{MyDatabaseLike}~s~d~\mathit{MyDatabase} \Rightarrow \\
    & &  \mathit{MyDatabaseStart}~\{
    \end{array}}  \\
\multicolumn{3}{l}{\begin{array}{lcl}
    \_\mathit{MyDatabase}\_\mathit{data} & :: & \mathit{SqlDatabase},\\
    \_\mathit{MyDatabase}\_\mathit{sub}  & :: & s
    \end{array}}\\
\} &&
\end{array}
\end{displaymath}
A $\mathit{MyDatabase}$ object has two possible states: either it is an instance of $\mathit{MyDatabase}$ or it is an instance of a sub-class of $\mathit{MyDatabase}$ which has been cast to $\mathit{MyDatabase}$. The $\mathit{MyDatabaseData}$ and $\mathit{MyDatabaseStart}$ constructors represent these cases respectively, if this $\Delta$-object is in the view. If it is not, then the $\mathit{MyDatabaseData}$ constructor may also be used to represent an ancestor to another $\Delta$-object. Figure \ref{tab:baseconstructors} shows how class attributes affect which constructors are required.
\begin{figure}
    \begin{center}
        \begin{tabular}{|c|c|c|c|}
            \hline \emph{Base class} & None & Abstract & Final \\ 
            \hline $\mathit{Data}$         & x & x & x \\ 
            \hline $\mathit{Start}$     & x & x & \\
            \hline
        \end{tabular} 
        \begin{tabular}{|c|c|c|c|}
            \hline \emph{Sub-class} & None & Abstract & Final \\ 
            \hline $\mathit{Data}$         & x &   & x \\ 
            \hline $\mathit{Start}$     & x & x &   \\ 
            \hline $\mathit{End}$     & x &   & x \\ 
            \hline $\mathit{Middle}$ & x & x &   \\ 
            \hline 
        \end{tabular} 
    \end{center}
    \caption{Object constructors by attribute}
    \label{tab:baseconstructors}
\end{figure}

Sub-class objects have more possible states because, unlike base class objects, they may have ancestors. For illustration, the encoding of $\mathit{DbModel}$ objects is given below:
\begin{displaymath}
\begin{array}{lcl}
\mathbf{data}~\mathit{DbModelObj} & = & \mathit{DbModelData}~\{ \\ 
\multicolumn{3}{l}{\quad \_\mathit{DbModel}\_\mathit{data} :: \mathit{Map}~\mathit{Int}~\mathit{Recipe}}\\
\multicolumn{3}{l}{\begin{array}{lcl}
    \} & \mid & \forall s~d.\mathit{DbModelLike}~s~d~\mathit{DbModel} \Rightarrow \\
    & &  \mathit{DbModelStart}~\{
    \end{array}}  \\
\multicolumn{3}{l}{\quad \begin{array}{lcl}
    \_\mathit{DbModel}\_\mathit{data} & :: & \mathit{Map}~\mathit{Int}~\mathit{Recipe},\\
    \_\mathit{DbModel}\_\mathit{sub}  & :: & s
    \end{array}}\\
\multicolumn{3}{l}{ \begin{array}{lcl}
    \} & \mid & \mathit{DbModelEnd}~\{ 
    \end{array}}  \\
\multicolumn{3}{l}{\quad \begin{array}{lcl}
    \_\mathit{DbModel}\_\mathit{sup}  & :: & \mathit{MyDatabase}, \\
    \_\mathit{DbModel}\_\mathit{data} & :: & \mathit{Map}~\mathit{Int}~\mathit{Recipe}
    \end{array}}\\
\multicolumn{3}{l}{ \begin{array}{lcl}
    \} & \mid & \forall s~d.\mathit{DbModelLike}~s~d~\mathit{DbModel} \Rightarrow \\
    & &  \mathit{DbModelMiddle}~\{
    \end{array}}  \\
\multicolumn{3}{l}{ \quad \begin{array}{lcl}
    \_\mathit{DbModel}\_\mathit{sup}  & :: & \mathit{MyDatabase}, \\
    \_\mathit{DbModel}\_\mathit{data} & :: & \mathit{Map}~\mathit{Int}~\mathit{Recipe}, \\
    \_\mathit{DbModel}\_\mathit{sub}  & :: & s
    \end{array}}\\
\} &&
\end{array}
\end{displaymath}

$\mathit{DbModel}$ is a sub-class that is neither abstract nor final. However, like $\mathit{MyDatabase}$, it can only be in one of two possible states while it is in view. If it is an instance of $\mathit{DbModel}$, then it will have an ancestor but no successors, which is represented by the $\mathit{DbModelEnd}$ constructor. If it is an instance of a sub-class of $\mathit{DbModel}$, then it does additionally have successors and the $\mathit{DbModelMiddle}$ constructor is used. Depending on whether a $\Delta$-object of this type is an instance of $\mathit{DbModel}$ or a sub-class, $\mathit{DbModelData}$ or $\mathit{DbModelStart}$ are used to represent it as a successor, respectively.

Instantiating new objects requires us to set up the initial zipper configuration. This is implemented in the $\mathit{new}$ function, which is a member of a type class so it can be overloaded for every non-abstract class (Section \ref{sec:th}). For example, to do this for $\mathit{DbModel}$ objects we have:
\begin{displaymath}
\begin{array}{lcl}
\mathit{new} & :: & (\mathit{SqlDatabase}, \mathit{Map}~\mathit{Int}~\mathit{Recipe}) \to 
\mathit{DbModelObj} \\
\mathit{new}~(v0,v1) & = & \mathit{DbModelEnd}~\{ \\
\multicolumn{3}{l}{\quad \begin{array}{lcl}
    \_\mathit{DbModel}\_\mathit{sup} & = & \mathit{new}~v0, \\
    \_\mathit{DbModel}\_\mathit{data} & = & v1
    \end{array}} \\
    \}&&
\end{array}
\end{displaymath}

\subsection{Type casts}

When an object is constructed, the view of the zipper is placed on the last element in the list -- the $\Delta$-object corresponding to the object class which is being instantiated. Its immediate ancestor is a $\Delta$-object corresponding to its superclass, which in turn may have its superclass as an ancestor, and so on. Such an object can be cast to a superclass by shifting the zipper's view to an ancestor, in which case the sub-class becomes a successor element (Figure \ref{fig:cast}). In this case, the type of the $\Delta$-object representing the sub-class is existentially-quantified and we no longer know anything about it, other than that it implements the same type class as the parent. For this reason, the zipper must first be traversed to construct the state of the whole object. it is therefore beneficial to be able to do this in layers using multiple state monad transformers.

\begin{figure}
    \begin{center}
        \bgroup
        \def\arraystretch{1.5}
        \begin{tabular}{ccc}
            \begin{tikzpicture}[node distance=2.0cm,auto,>=latex']
            \node [int] (a) {$\mathit{MyDatabase}$};
            \node [int,below=1cm,pin={[init]above:View}] (b) [right of=a] {$\mathit{DbModel}$};
            \path[->] (b) edge node {} (a);
            \end{tikzpicture} & $\qquad \Leftrightarrow \qquad$ &
            
            \begin{tikzpicture}[node distance=2.0cm,auto,>=latex']
            \node [int,pin={[init]above:View}] (c)  {$\mathit{MyDatabase}$};
            \node [int,above=1cm] (d) [right of=c] {$\mathit{DbModel}$};
            \path[->] (c) edge node {} (d);
            \end{tikzpicture} \\
            $\mathit{DbModel}$ object & & $\mathit{MyDatabase}$ object
        \end{tabular}
        \egroup
    \end{center}
    \caption{Cast between an $\mathit{DbModel}$ object and an $\mathit{MyDatabase}$ object} \label{fig:cast}
\end{figure}

The implementation of the illustrated cast is given below. This $\mathit{cast}$ function is part of a type class for which there are instances for every pair of sub-class and ancestor.
\begin{displaymath}
\begin{array}{lcl}
\multicolumn{3}{l}{\mathit{cast} :: \mathit{DbModelObj} \to \mathit{MyDatabaseObj}} \\
\mathit{cast}~(\mathit{DbModelEnd}~(\mathit{MyDatabaseData}~\mathit{pd})~\mathit{d}) & = & \\ \multicolumn{3}{l}{\quad \mathit{MyDatabaseStart}~\mathit{pd}~(\mathit{DbModelData}~d)} \\
\mathit{cast}~(\mathit{DbModelMiddle}~(\mathit{MyDatabaseData}~\mathit{pd})~\mathit{d}~\mathit{sub}) & = & \\ \multicolumn{3}{l}{\quad \mathit{MyDatabaseStart}~\mathit{pd}~(\mathit{DbModelStart}~d~\mathit{sub})}
\end{array}
\end{displaymath}
Casts in the other direction require a bit more work. Firstly, we need a different casting function whose result is wrapped in $\mathit{Maybe}$ to indicate that a cast from a base-class to a sub-class may not succeed. However, due to the use of existential types, we also require runtime type information about the type of the value in \emph{e.g.} the $\_\mathit{MyDatabase}\_\mathit{sub}$ field. Haskell's $\mathit{Typeable}$ type class\footnote{\url{https://hackage.haskell.org/package/base/docs/Data-Typeable.html}} may be used for this purpose.

\section{Translation}
\label{sec:auto}

Encoding object classes for our system by hand is both repetitive and error-prone. Luckily, the encoding is also well-suited for auto-generation using Template Haskell.
In this section, we propose syntactic sugar resembling the usual notation for object classes used by object-oriented languages. The corresponding encodings can automatically be generated from this syntax. Since the $\mathbf{class}$ keyword is already used for type classes in Haskell, object classes use the $\mathbf{state}$ keyword instead.
Declarations for object classes have syntax:
\begin{displaymath}
\begin{array}{l}
[\mathbf{abstract} \mid \mathbf{final}]~\mathbf{state}~\mathit{C}~\overline{\mathit{tyvar}}~[: P~\overline{\mathit{type}}]~\mathbf{where} \\
\quad \overline{\mathbf{data}~\mathit{var}~[= \mathit{expr}] :: \mathit{type}} \\
\quad \overline{[\mathbf{abstract}]~\mathit{var} :: \mathit{type}} \\
\quad \overline{\mathit{equation}}
\end{array}
\end{displaymath}
We use the usual overline and square-bracket meta-syntax for repetition
and optional elements respectively.
The $\mathbf{state}$ keyword is optionally preceded by either an $\mathbf{abstract}$ or $\mathbf{final}$ modifier, but cannot have both. If the $\mathbf{abstract}$ modifier is specified, the class cannot be instantiated directly, but it may include method typings marked as $\mathbf{abstract}$ which do not have an accompanying binding. Unless a class which inherits from an abstract class is abstract itself, it must implement all of its ancestors' methods which are marked as $\mathbf{abstract}$ and are therefore lacking an implementation. If a class is marked as $\mathbf{final}$, then no other class can inherit from it. This is accomplished by omitting the data constructors which allow successor $\Delta$-objects as shown in Figure \ref{tab:baseconstructors}.

The name of an object class $C$ follows the conventions for type constructors\footnote{In Haskell, this means that the name must start with an upper-case character.} and is followed by zero or more type variables. The optional $: \mathit{P}~\overline{\mathit{type}}$ component of the header of the declaration is used to specify the class's parent. $P$ must be the name of a non-final object class. There must be as many $\mathit{type}$s as there are type parameters in $P$'s definition. Any type variables used here must first be declared as type parameters of $C$.

The body of an object class declaration following the $\mathbf{where}$ keyword consists of three types of definitions which may appear in any order. A line starting with the $\mathbf{data}$ keyword is used to declare a field and must consist of at least a name and a type, but may optionally have an expression describing its default value. If no default value is specified, then it is assumed to be $\bot$. The fields in the object class declaration are desugared into fields for the state date type: 
\begin{displaymath}
\begin{array}{lcl}
\mathbf{data}~\mathit{CState}~\overline{\mathit{tyvar}} & = & \mathit{MkCState}~\{~\overline{\_\mathit{var} :: \mathit{type}}~\}
\end{array}
\end{displaymath}
The names of this data type and its single constructor can be chosen arbitrarily, since they are never exposed directly to the user. However, for consistency, we have used the name of the object class followed by $\mathit{State}$ for the name of the data type and $\mathit{Mk}$, followed by the name of the data type, for the constructor. This data type is parametrised over the same type parameters as $C$. The type alias for the state monad transformer parametrised over $\mathit{CState}$ follows:
\begin{displaymath}
\mathbf{type}~\mathit{C}~\overline{\mathit{tyvar}} = \mathit{StateT}~(\mathit{CState}~\overline{\mathit{tyvar}})~[\mathit{Identity} \mid PM~\overline{\mathit{type}}]
\end{displaymath}
By convention, the type alias is named after the object class. As with the state data type before, its type parameters are the same as in the header of the declaration for $C$. If the class has a parent, its underlying monad is that corresponding to $P$ applied to the type argument specified in the header of $C$. Otherwise, it is $\mathit{Identity}$. The last new type that results from a declaration for an object class $C$ is the data type representing objects of type $C$:
\begin{displaymath}
\mathbf{data}~\mathit{C}~\overline{\mathit{tyvar}} = \mathit{constructors}
\end{displaymath}
We refer to Figure \ref{tab:baseconstructors} for guidance on which constructors are depending on whether $C$ is a sub-class or not and which attributes it has. The $\mathit{Middle}$ constructor combines all possible fields which may be required by an object's internal state:
\begin{displaymath}
\begin{array}{l}
\forall s~d.\mathit{CLike}~s~d~(\mathit{C}~\overline{\mathit{tyvar}})~\overline{\mathit{tyvar}} \Rightarrow \mathit{CMiddle}~\{\\
\quad \begin{array}{lcl}
\_\mathit{C}\_\mathit{sup}  & :: & \mathit{PObject}~\overline{\mathit{type}}, \\
\_\mathit{C}\_\mathit{data} & :: & \mathit{CState}~\overline{\mathit{tyvar}}, \\
\_\mathit{C}\_\mathit{sub}  & :: & s~\overline{\mathit{tyvar}}
\end{array}\\
\}
\end{array}
\end{displaymath}
The existentially-quantified type variables $s$ and $d$ as well as the type class constraint are only required if the $\mathit{sub}$ field is present. $s$ refers to the object type of a sub-class and $d$ refers to the corresponding state type. The constraint enforces that the sub-class must implement all methods of $C$. The $\mathit{sup}$ field is only present in sub-classes, but never base classes. Its value is a $\Delta$-object belonging to $P$. Finally, the $\mathit{data}$ field is always present and stores the $\Delta$-object's state.

Method typings and fields are used to construct the type class representing $C$'s interface: 
\begin{displaymath}
\begin{array}{l}
\mathbf{class}~[\mathit{Monad}~m \mid \mathit{PLike}~o~s~m~\overline{\mathit{type}}] \Rightarrow \\
\quad \mathit{CLike}~o~s~m~\overline{\mathit{tyvar}} \mid o \to s, s \to o~\mathbf{where} \\
\qquad \begin{array}{l}
\_\mathit{C}\_\mathit{invoke} :: \mathit{CLike}~o'~d'~(\mathit{StateT}~(s~\overline{\mathit{tyvar}})~m)~\overline{\mathit{tyvar}} \Rightarrow \\
\qquad o'~\overline{\mathit{tyvar}} \to \\ \qquad (o'~\overline{\mathit{tyvar}} \to \mathit{StateT}~(s~\overline{\mathit{tyvar}})~m~(r,o'~\overline{\mathit{tyvar}})) \to\\\qquad
o~\overline{\mathit{tyvar}} \to m~(r,o~\overline{\mathit{tyvar}},o'~\overline{\mathit{tyvar}}) \\
\overline{\_\mathit{ext}\_\mathit{get}\_\mathit{field} :: o~\overline{\mathit{tyvar}} \to m~(\mathit{type}, o~\overline{\mathit{tyvar}}) } \\
\overline{\_\mathit{int}\_\mathit{get}\_\mathit{field} :: \mathit{StateT}~(s~\overline{\mathit{tyvar}})~m~\mathit{type}} \\
\overline{\_\mathit{ext}\_\mathit{set}\_\mathit{field} :: o~\overline{\mathit{tyvar}} \to \mathit{type} \to m~((),o~\overline{\mathit{tyvar}})} \\
\overline{\_\mathit{int}\_\mathit{set}\_\mathit{field} :: \mathit{type} \to \mathit{StateT}~(s~\overline{\mathit{tyvar}})~m~()} \\
\overline{\_\mathit{ext\_method} :: o~\overline{\mathit{tyvar}} \to \overline{\mathit{arg} \to}~m~(\mathit{result},o~\overline{\mathit{tyvar}})}\\
\overline{\_\mathit{int\_method} :: \overline{\mathit{arg} \to}~ \mathit{StateT}~(s~\overline{\mathit{tyvar}})~m~\mathit{result} }
\end{array}
\end{array}
\end{displaymath}

For each field, there is a getter and a setter. Just like methods, each has an internal and an external version. Note that we treat getters and setters as state transformers, just like methods. We take the view that they are \emph{properties} as they are found in \emph{e.g.} C\#. That is, while we have the expectation that a getter gets the value of a field and that a setter sets the value of the field, both can also potentially be implemented as arbitrary methods with matching types. Additionally, the $\_\mathit{C}\_\mathit{invoke}$ method is used by sub-classes to invoke their own methods in a monad stack set up by $C$.

An instance of this type class is generated as well as one for each type class belonging to an ancestor of $C$. Each external method's implementation pattern matches on the configuration of the object, adds a state monad transformer layer on top of the existing monad stack, and then either invokes its own, internal implementation of the method or propagates the call to a sub-class.

If $C$ is not a base class, a second instance of $\mathit{CLike}$ is generated with the underlying monad set to $\mathit{Identity}$. In this case, all methods use $\_P\_\mathit{invoke}$ on the $\Delta$-object belonging to $P$ to perform the process shown in Figure \ref{fig:invoke}:
\begin{displaymath}
\begin{array}{l}
\mathbf{instance}~\mathit{CLike}~\mathit{C}~\mathit{CState}~\mathit{Identity}~\overline{\mathit{tyvar}}~\mathbf{where}\\
\quad \begin{array}{lcl}
\mathit{method}~(\mathit{CEnd}~\mathit{sup}~\mathit{d}) & = & \mathbf{do} \\
\multicolumn{3}{l}{\quad \begin{array}{lcl}
    (r,sup',\mathit{CEnd}~\_~d') & \leftarrow & \\ 
    \multicolumn{3}{l}{ \quad \_P\_\mathit{invoke}~(\mathit{CEnd}~\mathit{sup}~d)~\mathit{method}~\mathit{sup} }\\
    \multicolumn{3}{l}{\mathit{return}~(r, \mathit{CEnd}~sup'~d')}
    \end{array}}
\end{array}
\end{array}
\end{displaymath}
%Methods are mostly standard Haskell. $<:$ and $>:$ statements are desugared into calls to the appropriate getters or setters for a field. Method invocations (identifiers containing dots) are rewritten so that the object becomes the first argument of the method and the resulting expression is then wrapped into $\mathit{runIdentity}$.
